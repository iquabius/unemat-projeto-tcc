\documentclass[
  12pt,         % tamanho da fonte
  a4paper,      % tamanho do papel.
  oneside,
  sumario=tradicional,% sumário padrão latex
  chapter=TITLE,      % títulos de capítulos convertidos em letras maiúsculas
  section=TITLE,
  english,      % idioma adicional para hifenização
  brazil,       % o último idioma é o principal do documento
  ]{abntex2}

% ---
% PACOTES
% ---

% ---
% Pacotes fundamentais
% ---
\usepackage{lmodern}      % Usa a fonte Latin Modern
\usepackage[T1]{fontenc}    % Selecao de codigos de fonte.
\usepackage[utf8]{inputenc}    % Codificacao do documento (conversão automática dos acentos)
\usepackage{indentfirst}    % Indenta o primeiro parágrafo de cada seção.
\usepackage{color}        % Controle das cores
\usepackage{graphicx}      % Inclusão de gráficos
\usepackage{microtype}       % para melhorias de justificação
% \usepackage{enumitem}
% \usepackage[normalem]{ulem}
% \usepackage{makeidx}
% \usepackage{amsmath}
\usepackage{unemat}

% pacotes relacionados a tabelas
\usepackage{amsfonts} % checkmark symbol
\usepackage{multirow} % expande uma célula pra várias linhas
\usepackage{diagbox}  % usado para desenhar um linha diagonal numa célula
\usepackage[export]{adjustbox} % centraliza tabelas/figuras além das margens do texto
% ---

% ---
% Pacotes de citações
% ---
\usepackage[brazilian,hyperpageref]{backref}
% "num" = numéricas; alf = autor-data
\usepackage[alf,abnt-etal-list=0]{abntex2cite}

% ---
% CONFIGURAÇÕES DE PACOTES
% ---

% ---
% Configurações do pacote backref
% Usado sem a opção hyperpageref de backref
\renewcommand{\backrefpagesname}{Citado na(s) página(s):~}
% Texto padrão antes do número das páginas
\renewcommand{\backref}{}
% Define os textos da citação
\renewcommand*{\backrefalt}[4]{
  \ifcase #1 %
    Nenhuma citação no texto.%
  \or
    Citado na página #2.%
  \else
    Citado #1 vezes nas páginas #2.%
  \fi}%
% ---

% ---
% Informações de dados para CAPA e FOLHA DE ROSTO
% ---
\titulo{Coordenação de Eventos em Interfaces
        Gráficas de Aplicações Web com
        Programação Funcional Reativa}
\autor{Josias Duarte Busiquia}
\local{Barra do Bugres -- MT}
\data{2016}
\instituicao{%
  Universidade do Estado de Mato Grosso - UNEMAT
  \par
  Faculdade de Ciências Exatas e Tecnológicas - FACET
  \par
  Departamento de Ciência da Computação}
\tipotrabalho{Monografia}
% O preambulo deve conter o tipo do trabalho, o objetivo,
% o nome da instituição e a área de concentração
\preambulo{Trabalho de conclusão de curso apresentado ao curso de Ciência
da Computação da Universidade do Estado de Mato Grosso – UNEMAT, como
requisito parcial para obtenção do grau de Bacharel, sob orientação do
Professor MSc. Luciano Zamperetti Wolski.}
% ---

% ---
% Configurações de aparência do PDF final

% alterando o aspecto da cor azul
\definecolor{blue}{RGB}{41,5,195}

% informações do PDF
\makeatletter
\hypersetup{
  %pagebackref=true,
  pdftitle={\@title},
  pdfauthor={\@author},
  pdfsubject={\imprimirpreambulo},
  pdfcreator={LaTeX with abnTeX2},
  pdfkeywords={abnt}{latex}{abntex}{abntex2}{projeto de pesquisa},
  colorlinks=true,           % false: boxed links; true: colored links
  linkcolor=black,            % color of internal links
  citecolor=black,            % color of links to bibliography
  filecolor=black,          % color of file links
  urlcolor=black,
  bookmarksdepth=4
}
\makeatother
% ---

% ---
% Espaçamentos entre linhas e parágrafos
% ---

% O tamanho do parágrafo é dado por:
\setlength{\parindent}{1.3cm}

% Controle do espaçamento entre um parágrafo e outro:
\setlength{\parskip}{0.2cm}  % tente também \onelineskip

\makeindex

\begin{document}

% Seleciona o idioma do documento (conforme pacotes do babel)
\selectlanguage{brazil}

% Retira espaço extra obsoleto entre as frases.
\frenchspacing

% ----------------------------------------------------------
% ELEMENTOS PRÉ-TEXTUAIS (Preamble)
% ----------------------------------------------------------
% \pretextual

\imprimircapa
\imprimirfolhaderosto
% ---

% ---
% inserir lista de ilustrações
% ---
%\pdfbookmark[0]{\listfigurename}{lof}
%\listoffigures*
%\cleardoublepage
% ---

% ---
% inserir lista de tabelas
% ---
%\pdfbookmark[0]{\listtablename}{lot}
%\listoftables*
%\cleardoublepage
% ---

% ---
% inserir lista de abreviaturas e siglas
% ---
\begin{siglas}
  \item[FRP] \emph{Functional Reactive Programming}
  \item[HTML5] \emph{HyperText Markup Language}, versão 5
\end{siglas}


% ---
% inserir o sumario
% ---
\pdfbookmark[0]{\contentsname}{toc}
{\center\tableofcontents*}
\cleardoublepage
% ---

% ----------------------------------------------------------
% ELEMENTOS TEXTUAIS
% ----------------------------------------------------------
\textual

% ----------------------------------------------------------
% Capitulo de textual
% ----------------------------------------------------------

% \include força uma quebra de página
% \chapter{Projeto de Pesquisa}
\chapter*{Projeto de Pesquisa}
\markboth{Projeto de Pesquisa}{Projeto de Pesquisa}
\addcontentsline{toc}{chapter}{Projeto de Pesquisa}
\section{Tema}\label{ltema}

Programação de aplicações orientadas a eventos.

\section{Delimitação do tema}\label{ldelimitacao}

Coordenação de eventos na programação de Interfaces Gráficas
do Usuário (GUIs)\footnote{
  Para manter interação contínua com o ambiente externo, GUIs
  precisam reagir a vários eventos, como cliques do mouse ou
  pressionamento de teclas.
  Tais eventos são processados para
  executar uma tarefa correspondente, como atualizar o estado interno
  da aplicação ou exibir dados na tela.
  Devido a essas propriedades, uma GUI é considerada um sistema
  \emph{reativo} ou \emph{orientado a eventos.}
} com Programação Funcional Reativa (PFR).

\section{Formulação do problema}\label{lproblema}

\subsection{Variáveis}

\begin{itemize}[noitemsep]
  \item Aplicações \textit{web} interativas
  \item \textit{Programação Assíncrona}
  \item Coordenação de eventos
    \begin{itemize}[noitemsep]
      \item Internos: \textit{mouse}, teclado, \textit{touchscreen}, etc.
      \item Externos: mensagens de servidores remotos.
    \end{itemize}
  \item \textit{Programação Funcional Reativa}
\end{itemize}


\subsection{Pergunta}

O quê o paradigma de \textit{Programação Funcional Reativa}
pode oferecer para lidar com a complexidade de se coordenar
eventos assíncronos em aplicações \textit{web} interativas?

% Variáveis
% - produtividade no desenvolvimento
% - coesão da base de código
% - expressividade do código
% - composicionalidade (compositionality)
% - previsibilidade do sistema
% - testabilidade
% - easier to reason about
% - testes de software
% - raciocínio informal (informal reasoning)
% - efeitos colaterais (side effects)

\section{Hipóteses}\label{lhipoteses}


\subsection{Hipótese Básica}

Interfaces interativas em aplicações web
são inerentemente orientadas a eventos assíncronos,
ou seja, precisam reagir a comandos do usuário e
mensagens de servidores remotos.
Recentemente o paradigma \textit{FRP}
tem chamado a atenção de
comunidades de desenvolvimento web, \textit{mobile}
e \textit{desktop}, devido as facilidades que \textit{FRP}
oferece para se programar aplicações orientadas a eventos.
Conceitos de \textit{FRP} podem ser usados
na programação de interfaces gráficas interativas
-- que são altamente orientadas a eventos --
e oferecem uma maneira de coordenar
computações assíncronas\footnote{
  Clicks do \textit{mouse}, pressionamento de teclas no teclado,
  requisições ao servidor, etc.
}
através da composição de eventos de forma declarativa
e concisa, resultando em código expressivo e de fácil
manutenção.


\subsection{Hipóteses Secundárias}

\begin{itemize}[noitemsep]
  \item \textit{FRP} fornece um modelo de programação com um
        nível mais elevado de abstração.
  \item O paradigma \textit{FRP} pode ser difícil de ser adotado
        devido ao alto nível de abstração.
\end{itemize}

% Hipóteses
% - PF torna um sistema mais previsível através do gerenciamento do estado
% - PF oferece melhor reuso de código através da composicionalidade

\section{Objetivos}\label{lobjetivos}

\subsection{Objetivo Geral}

Demonstrar e analisar a PFR em comparação a modelos baseados em
\emph{callbacks}, quanto ao nível de abstração fornecido à coordenação de
eventos na programação de GUIs.

\subsection{Objetivos Específicos}

\begin{itemize}[noitemsep]
  \item Demonstrar o modelo \emph{declarativo} da
        \emph{Programação Funcional}, e o modelo tradicional \emph{imperativo},
        aplicados na manipulação de sequências (e.g. \emph{arrays}, listas,
        mapas);
  \item Demonstrar o modelo \emph{declarativo} da PFR,
        e o modelo tradicional \emph{imperativo} baseado
        \emph{callbacks}, aplicados na coordenação
        de eventos em GUIs;
  \item Analisar o modelo \emph{declarativo} em contraste com o
        \emph{imperativo} no que concerne o nível de abstração
        fornecido à manipulação de sequências,
        e à coordenação de eventos;
  \item Verificar as abordagens visando a compreensibilidade
        do programa \emph{declarativo} e \emph{imperativo;}
\end{itemize}

%%% Local Variables:
%%% mode: latex
%%% TeX-master: "../projeto"
%%% End:

\section{Justificativa}\label{ljustificativa}

% Aplicações modernas|contemporâneas estão se tornando cada vez mais interativas,
% precisando reagir|responder a muitos estímulos/eventos imprevisíveis do ambiente
% externo -- como comandos de um usuário, sinais de um sensor, ou requisições|
% menssagens em uma rede.
% Uma interfaces gráfica precisa responder a comandos do usuário, um
% sistema embarcado reage a sinais de sensores, e um sistema distribuído precisa
% processar requisições em uma rede.

% - Mencionar que o javascript foi inventado para tornar as páginas interativas
% - Interfaces gráficas (GUIs) são inerentemente orientadas a eventos (event-driven)
% - GUIs possuem a característica de precisarem reagir e coordernar vários eventos,
%   como clicks do mouse, pressionamento de teclas, gestos no touchscreen, etc.
% - Alto grau de responsividade
% - websockets, push notifications, AppCache, service workers, web workers
%   - http://bit.ly/serviceworkers_webworkers_websockets
Interatividade em páginas \textit{web} se deu com a introdução do
\textit{JavaScript} em navegadores
\cite{
  paolini1995netscape,
  thau2000javascript}.
O advento de outras tecnologias tem tornado as interfaces \textit{web}
cada vez mais interativas
(e.g. Ajax\footciteref{garrett2005ajax} e \textit{Web Sockets\footnotemark}),
\footnotetext{Tecnologia recentemente definida pela especificação do \gls{html5}}
dando origem a uma nova gama de aplicações \textit{web} com interfaces ricas que oferecem ao usuário uma
experiência similar as aplicações \textit{mobile} ou \textit{desktop}.
Assim como em qualquer interface gráfica, interfaces \textit{web} precisam reagir
a vários eventos imprevisíveis do ambiente externo,
provindos tanto do usuário (e.g. \textit{clicks} do mouse, pressionamento de teclas, etc) % ponto dps de etc?
quanto de outro software (e.g. mensagens do servidor). % ponto antes do parenteses?
% pois essa precisa reagir a vários eventos imprevisíveis do ambiente externo,
% provindos tanto do usuário (e.g. \textit{clicks} do mouse, pressionamento de teclas, etc)
% quanto de outro software (e.g. mensagens do servidor).

% A coordenação de desses eventos é o aspecto mais complexo de um programa interativo.
%  in a timely fashion = em tempo hábil
Atualmente o modelo de programação mais empregado na coordenação desses eventos
em programas interativos é o \textit{event-driven programming}\footnotemark,
que consiste de um \textit{event-loop} que espera por eventos de forma contínua,
e quando um evento é detectado, uma função de
\textit{callback} apropriada é chamada para tratá-lo.
\footnotetext{Programação orientada a eventos.}
Essa abordagem configura uma das formas mais complexas de se programar sistemas interativos
\cite{edwards2009coherent,maier2010deprecating,reppy1992higher}, devido ao fato de que
aplicações desenvolvidas utilizando esse mecanismo apresentam um fluxo de controle
desestruturado e imprevisível, além de depender crucialmente de \textit{efeitos colaterais\footnotemark} pra
gerenciar seu estado \cite{meyerovich2009flapjax,muller2015interactive,muller2015practical}.
\footnotetext{Do inglês \textit{side-effects:} característica muito comum em
linguagens imperativas, onde uma função ou expressão pode modificar algum estado
externo (e.g. alterar uma variável global, produzir uma saída na tela/terminal,
escrever no sistema de arquivos, etc).
Em programação funcional o uso de efeitos colaterais é desencorajado, e deve ser
usado apenas quando absolutamente necessário -- e.g. manipular uma variável
global não é absolutamente necessário, mas imprimir uma mensagem na tela pode ser.}
Na literatura, essa abordagem é descrita como \textit{"Callback Hell"}, devido % colocar página na citação
a forma desconcertante com que o fluxo de controle coordena mudanças no estado
do programa \cite[p.~2]{edwards2009coherent}. % bainomugisha2013survey, muller2015practical
% inversão de controle

Vale ressaltar que a preocupação desnecessária com o fluxo de controle e o mau
gerenciamento de estado são consideradas as principais causas de complexidade em
sistemas contemporâneos, pois afetam o entendimento das várias partes do código
por parte do desenvolvedor, além de dificultar a realização de testes de software
\cite{Moseley06outof}.
Uma análise das aplicações \textit{desktop} da Adobe, relatada em 2006, indicou
que o código que coordena a lógica de manuseio de eventos, \textit{widgets},
e outros componentes da interface gráfica, representa cerca de um terço do código, e
% manuseio, manejo, manipulação
% que o código responsável por componentes da interface gráfica e da lógica de
% coordenação de eventos representa cerca de um terço do código, e
mais da metade dos \textit{bugs} reportados \cite{jarvi2008property}.
Sendo interfaces gráficas com alto grau de interatividade parte inerente
de uma aplicação, seu desenvolvimento e manutenção se tornam um desafio.

%   - observer pattern

% declarative vs imperative
%   - specification (what) vs. execution (how)
%     - Declarative Interaction Design for Data Visualization
%   - modeling vs presentation
%     - Elm
%     - FR Animation
%
% FRP
%  - Outros tipos de software podem ser considerados reativos, como um sistema
%    embarcado que reage a sinais de sensores, ou um sistema distribuído que
%    precisa reagir a mensagens na rede.

<Apresentar FRP como alternativa>

% - Documentar o estado da arte em:
%   - técnologias web
%   - programação assíncrona
% "Este trabalho tem por objetivo apresentar os conceitos, objetivos, tecnologias e
% demais questões envolvidas na abordagem de desenvolvimento de aplicações
% Web conhecida como Ajax. E prover uma aplicação Web de
% georeferenciamento do campus da UFSC utilizando a abordagem Ajax."

<Descrever proposta do projeto>

\section{Metodologia}\label{lmetodologia}

% Natureza: Aplicada
% Quanto aos Objetivos: Exploratória
% Pesquisa bibliográfica
% Quanto ao Procedimento: Estudo de Caso/casos de uso/casos de estudo
% Quanto a Abordagem: Qualitativa
% Coleta de dados: Observação direta

Apesar do paradigma \textit{FRP} ter sido apresentado há
quase duas décadas com o trabalho de \citeauthoronline{Elliott97franimation}
em \citeyear{Elliott97franimation}, sua exploração acadêmica
e aplicação por parte de pesquisadores e da indústria ainda é
recente. Por esse motivo este trabalho tem um objetivo de
cunho exploratório, que \citeauthoronline{gil2010metodos}
descreve da seguinte forma:

\begin{citacao}
  Pesquisas exploratórias são desenvolvidas com o
  objetivo de proporcionar visão geral, de tipo aproximativo,
  acerca de determinado fato.
  Este tipo de pesquisa é realizado especialmente quando o
  tema escolhido é pouco explorado e torna-se difícil sobre
  ele formular hipóteses precisas e operacionalizáveis
  (\citeyear{gil2010metodos}, p. 20).
\end{citacao}

A abordagem a ser utilizada será qualitativa, que não se
preocupa com a valores numéricos, mas procura aprofundar
a compreensão do objeto de estudo \cite[p.~31]{gerhardt2009metodos}.
A análise qualitativa será feita através de estudos de casos
implementados com as ferramentas a serem estudadas.
Para \citeauthoronline{santos2005manual}, um estudo de caso:

\begin{citacao}
  É o estudo que analisa com profundidade um ou poucos fatos,
  com vistas à obtenção de um grande conhecimento com riqueza
  de detalhes do objeto estudado. É usada nos estudos exploratórios
  e no início de pesquisas mas complexas. Tem aplicação em
  qualquer área do conhecimento (\citeyear{gil2010metodos}, p. 174).
\end{citacao}

\section{Cronograma}\label{lcronograma}

\begin{adjustbox}{center}
  \tiny
  \begin{tabular}{|p{4cm}|c|c|c|c|c|c|c|c|c|c|c|l|}
    \cline{1-12}
      \multicolumn{1}{|c|}{
        \multirow{2}{*}{
          \diagbox[width=4.4cm]{
            \textbf{Atividades}
          }{
            \textbf{Ano/Mês}
          }}}
      & \multicolumn{6}{c|}{\textbf{2016/1}}
      & \multicolumn{5}{c|}{\textbf{2016/2}} \\
      \cline{2-12}
      & Fev. & Mar. & Abr. & Maio & Jun. & Jul. & Ago. & Set. & Out. & Nov. & Dez. \\
    \hline
      Desenvolvimento do Tema e Objetivos
      & X &  &  &  &  &  &  &  &  &  &  \\
    \hline
      Desenvolvimento do Problema, Hipótese e Justificativa
      & & X & X & X &  &  &  &  &  &  &  \\
    \hline
      Desenvolvimento da Metodologia e Fundamentação Teórica
      & & & & X & X & X & & & & & \\
    \hline
      Desenvolvimento do Cronograma e Orçamento
      & & & X & & & & & & & & \\
    \hline
      Encontros de Orientação
      & & X & X & X & X & X & X & X & X & X & X \\
    \hline
      Defesa do Projeto de Pesquisa
      & & & & & X & & & & & & \\
    \hline
      Desenvolvimento da Introdução
      & & & & & & X & X & & & & \\
    \hline
      Desenvolvimento do TCC
      & & & & & X & X & X & X & X & X & \\
    \hline
      Desenvolvimento dos Casos de Uso
      & & & & & & & X & X & X & & \\
    \hline
      Correção de Erros
      & & X & X & X & X & X & X & X & X & X & X \\
    \hline
      Elaboração da apresentação do TCC
      & & & & & & & & & & X & X \\
    \hline
      Apresentação do TCC
      & & & & & & & & & & & X \\
    \hline
      Entrega da Versão final
      & & & & & & & & & & & X \\
    \hline
  \end{tabular}
\end{adjustbox}

\section{Orçamento}
\label{sec:orcamento}


\begin{center}
  \tiny
  \begin{tabular}{| l | r | r |}
    \hline
      \textbf{Descrição das Despesas} & \textbf{Quantidade} & \textbf{Valor Estimado} \\
    \hline
      Papel sulfite A4 500 folhas & 1 & R\$ 25,00 \\
    \hline
      Aquisição de cartucho e tinta & 1 & R\$ 100,00 \\
    \hline
      Encadernação & 4 & R\$ 30,00 \\
    \hline
      Aquisição de livros & 1 & R\$ 160,00 \\
    \hline
      Confecção do \emph{CD} e capa & 1 & R\$ 25,00 \\
    \hline
      \textbf{Total} & & \textbf{R\$ 340,00} \\
  \hline
  \end{tabular}
\end{center}

%%% Local Variables:
%%% mode: latex
%%% TeX-master: "../projeto"
%%% End:


% ---
% Finaliza a parte no bookmark do PDF
% para que se inicie o bookmark na raiz
% e adiciona espaço de parte no Sumário
% ---

\phantompart

% ----------------------------------------------------------
% ELEMENTOS PÓS-TEXTUAIS
% ----------------------------------------------------------
\postextual

% ----------------------------------------------------------
% Referências bibliográficas
% ----------------------------------------------------------
\bibliography{refs}

\phantompart
\printindex

\end{document}
