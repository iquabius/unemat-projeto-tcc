\section{Objetivos}\label{lobjetivos}

\subsection{Objetivo Geral}

Demonstrar e analisar a PFR em comparação a modelos baseados em
\emph{callbacks}, quanto ao nível de abstração fornecido à coordenação de
eventos na programação de GUIs.

\subsection{Objetivos Específicos}

\begin{itemize}[noitemsep]
  \item Demonstrar o modelo \emph{declarativo} da
        \emph{Programação Funcional}, e o modelo tradicional \emph{imperativo},
        aplicados na manipulação de sequências (e.g. \emph{arrays}, listas,
        mapas);
  \item Demonstrar o modelo \emph{declarativo} da PFR,
        e o modelo tradicional \emph{imperativo} baseado
        \emph{callbacks}, aplicados na coordenação
        de eventos em GUIs;
  \item Analisar o modelo \emph{declarativo} em contraste com o
        \emph{imperativo} no que concerne o nível de abstração
        fornecido à manipulação de sequências,
        e à coordenação de eventos;
  \item Verificar as abordagens visando a compreensibilidade
        do programa \emph{declarativo} e \emph{imperativo;}
\end{itemize}

%%% Local Variables:
%%% mode: latex
%%% TeX-master: "../projeto"
%%% End:
