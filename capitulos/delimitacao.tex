\section{Delimitação do tema}\label{ldelimitacao}

Coordenação de eventos na programação de Interfaces Gráficas
do Usuário (GUIs)\footnote{
  Para manter interação contínua com o ambiente externo, GUIs
  precisam reagir a vários eventos, como cliques do mouse ou
  pressionamento de teclas.
  Tais eventos são processados para
  executar uma tarefa correspondente, como atualizar o estado interno
  da aplicação ou exibir dados na tela.
  Devido a essas propriedades, uma GUI é considerada um sistema
  \emph{reativo} ou \emph{orientado a eventos.}
} com Programação Funcional Reativa (PFR).
