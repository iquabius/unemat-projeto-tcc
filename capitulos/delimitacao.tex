\section{Delimitação do tema}\label{ldelimitacao}

Coordenação de eventos\footnote{
  Um evento é uma ação ou ocorrência identificada por um
  programa, que pode ser tratado programaticamente.
  Eventos podem ser internos, provindos de dentro do próprio programa,
  ou externos, como os que originam-se de dispositivos de entrada manipulados
  pelo usuário (e.g. mouse, teclado, \emph{touchscreen}),
  sensores (e.g. acelerômetro, giroscópio),
  computadores remotos, ou outros programas/\emph{threads}.
} na programação de Interfaces Gráficas do Usuário (GUIs)\footnote{
  Para manter interação com o ambiente externo, GUIs
  precisam continuamente reagir a eventos, que são processados para
  executar uma tarefa correspondente, como atualizar o estado interno
  ou exibir dados.
  Devido a isso, uma GUI é considerada um sistema \emph{reativo}
  ou \emph{orientado a eventos.}
}.
