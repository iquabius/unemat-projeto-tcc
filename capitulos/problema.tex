\section{Problema}\label{lproblema}

O \emph{Observer Pattern}, usado na \emph{Programação Orientada
a Objetos (POO)}, é o modelo tradicionalmente predominante
na programação de GUIs.
No entanto, ele configura uma das formas mais complexas
de se programar, como é relatado por
\citeonline{maier2010deprecating},
\citeonline{edwards2009coherent},
\citeonline{fischer2007tasks},
e \citeonline{reppy1992higher}.
Recentemente o paradigma de \emph{Programação Funcional Reativa (PFR)},
que tem raízes na \emph{Programação Funcional (PF)},
tem sido explorado como uma alternativa promissora.
Posto isso, o que a PFR tem a oferecer, e como ela se compara ao
\emph{Observer Pattern} quanto ao nível de abstração fornecido para
coordenação de eventos na programação de GUIs?

% The observer pattern violates an impressive line-up of important
% software engineering principles, like "side-effects, encapsulation,
% composibility, separation of concerns, scalability, uniformity,
% abstraction, resource managment, and semantic distance.
% ~ Maier (2010), Deprecating the Observer Pattern
%
% "Unfortunately event-driven programming can create more coordination
% problems than it solves." ~ Edwards (2009), Coherent Reaction
%
% "Unfortunately, programming with events comes at a cost:
% event-driven programs are extremely difficult to understand
% and maintain." ~ Fischer (2007), Tasks: Language Support for
% Event-driven Programming
%
% "This structure is a poor-man's concurrency: the event-handlers are
% coroutines and the event-loop is the scheduler." ~ Reppy (1992),
% Higher-order concurrency

%%% Local Variables:
%%% mode: latex
%%% TeX-master: "../projeto"
%%% End:
