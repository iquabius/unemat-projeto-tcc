\section{Hipóteses}\label{lhipoteses}


\subsection{Hipótese Básica}

Interfaces interativas em aplicações web
são inerentemente orientadas a eventos assíncronos,
ou seja, precisam reagir a comandos do usuário e
mensagens de servidores remotos.
Recentemente o paradigma \textit{FRP}
tem chamado a atenção de
comunidades de desenvolvimento web, \textit{mobile}
e \textit{desktop}, devido as facilidades que \textit{FRP}
oferece para se programar aplicações orientadas a eventos.
Conceitos de \textit{FRP} podem ser usados
na programação de interfaces gráficas interativas
-- que são altamente orientadas a eventos --
e oferecem uma maneira de coordenar
computações assíncronas\footnote{
  Clicks do \textit{mouse}, pressionamento de teclas no teclado,
  requisições ao servidor, etc.
}
através da composição de eventos de forma declarativa
e concisa, resultando em código expressivo e de fácil
manutenção.


\subsection{Hipóteses Secundárias}

\begin{itemize}[noitemsep]
  \item \textit{FRP} fornece um modelo de programação com um
        nível mais elevado de abstração.
  \item O paradigma \textit{FRP} pode ser difícil de ser adotado
        devido ao alto nível de abstração.
\end{itemize}

% Hipóteses
% - PF torna um sistema mais previsível através do gerenciamento do estado
% - PF oferece melhor reuso de código através da composicionalidade
