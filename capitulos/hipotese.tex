\section{Hipóteses}\label{lhipoteses}


\subsection{Hipótese Básica}

Interfaces interativas em aplicações \textit{web}
são inerentemente orientadas a eventos assíncronos
ou seja, precisam reagir a comandos do usuário e
mensagens de servidores remotos.
Recentemente o paradigma de \textit{Programação Funcional Reativa}
tem chamado a atenção de
comunidades de desenvolvimento \textit{web}, \textit{mobile}
e \textit{desktop}, devido as facilidades que essas
oferecem para se desenvolver programas orientados a eventos.
Conceitos de ambos os paradigmas podem ser usados
no desenvolvimento de interfaces gráficas,
e oferecem uma maneira de coordenar
computações assíncronas\footnote{
  Clicks do mouse, pressionamento de teclas no teclado,
  requisições ao servidor, etc.
}
através da composição de eventos de forma declarativa
e concisa, resultando em código expressivo e de fácil
manutenção.


\subsection{Hipóteses Secundárias}

\textit{Programação Funcional Reativa}:
\begin{itemize}[noitemsep]
  \item Fornecem um modelo de programação com um nível mais elevado
        de abstração.
  \item Podem ser difíceis de serem adotados devido ao nível de abstração
        mais elevado.
\end{itemize}

% Hipóteses
% - PF torna um sistema mais previsível através do gerenciamento do estado
% - PF oferece melhor reuso de código através da composicionalidade
