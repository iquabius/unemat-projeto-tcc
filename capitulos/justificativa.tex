\section{Justificativa}\label{ljustificativa}

% Aplicações modernas|contemporâneas estão se tornando cada vez mais interativas,
% precisando reagir|responder a muitos estímulos/eventos imprevisíveis do ambiente
% externo -- como comandos de um usuário, sinais de um sensor, ou requisições|
% menssagens em uma rede.
% Uma interfaces gráfica precisa responder a comandos do usuário, um
% sistema embarcado reage a sinais de sensores, e um sistema distribuído precisa
% processar requisições em uma rede.

% - Mencionar que o javascript foi inventado para tornar as páginas interativas
% - Interfaces gráficas (GUIs) são inerentemente orientadas a eventos (event-driven)
% - GUIs possuem a característica de precisarem reagir e coordernar vários eventos,
%   como clicks do mouse, pressionamento de teclas, gestos no touchscreen, etc.
% - Alto grau de responsividade
% - websockets, push notifications, AppCache, service workers, web workers
%   - http://bit.ly/serviceworkers_webworkers_websockets
Interatividade em páginas \textit{web} se deu com a introdução do
\textit{JavaScript} em navegadores
\cite{
  paolini1995netscape,
  thau2000javascript}.
O advento de outras tecnologias tem tornado as interfaces \textit{web}
cada vez mais interativas
(e.g. Ajax\footciteref{garrett2005ajax} e \textit{Web Sockets\footnotemark}),
\footnotetext{Tecnologia recentemente definida pela especificação do \gls{html5}}
dando origem a uma nova gama de aplicações \textit{web} com interfaces ricas que oferecem ao usuário uma
experiência similar as aplicações \textit{mobile} ou \textit{desktop}.
Assim como em qualquer interface gráfica, interfaces \textit{web} precisam reagir
a vários eventos imprevisíveis do ambiente externo,
provindos tanto do usuário (e.g. \textit{clicks} do mouse, pressionamento de teclas, etc) % ponto dps de etc?
quanto de outro software (e.g. mensagens do servidor). % ponto antes do parenteses?
% pois essa precisa reagir a vários eventos imprevisíveis do ambiente externo,
% provindos tanto do usuário (e.g. \textit{clicks} do mouse, pressionamento de teclas, etc)
% quanto de outro software (e.g. mensagens do servidor).

% A coordenação de desses eventos é o aspecto mais complexo de um programa interativo.
%  in a timely fashion = em tempo hábil
Atualmente o modelo de programação mais empregado na coordenação desses eventos
em programas interativos é o \textit{event-driven programming}\footnotemark,
que consiste de um \textit{event-loop} que espera por eventos de forma contínua,
e quando um evento é detectado, uma função de
\textit{callback} apropriada é chamada para tratá-lo.
\footnotetext{Programação orientada a eventos.}
Essa abordagem configura uma das formas mais complexas de se programar sistemas interativos
\cite{edwards2009coherent,maier2010deprecating,reppy1992higher}, devido ao fato de que
aplicações desenvolvidas utilizando esse mecanismo apresentam um fluxo de controle
desestruturado e imprevisível, além de depender crucialmente de \textit{efeitos colaterais\footnotemark} pra
gerenciar seu estado \cite{meyerovich2009flapjax,muller2015interactive,muller2015practical}.
\footnotetext{Do inglês \textit{side-effects:} característica muito comum em
linguagens imperativas, onde uma função ou expressão pode modificar algum estado
externo (e.g. alterar uma variável global, produzir uma saída na tela/terminal,
escrever no sistema de arquivos, etc).
Em programação funcional o uso de efeitos colaterais é desencorajado, e deve ser
usado apenas quando absolutamente necessário -- e.g. manipular uma variável
global não é absolutamente necessário, mas imprimir uma mensagem na tela pode ser.}
Na literatura, essa abordagem é descrita como \textit{"Callback Hell"}, devido % colocar página na citação
a forma desconcertante com que o fluxo de controle coordena mudanças no estado
do programa \cite[p.~2]{edwards2009coherent}. % bainomugisha2013survey, muller2015practical
% inversão de controle

Vale ressaltar que a preocupação desnecessária com o fluxo de controle e o mau
gerenciamento de estado são consideradas as principais causas de complexidade em
sistemas contemporâneos, pois afetam o entendimento das várias partes do código
por parte do desenvolvedor, além de dificultar a realização de testes de software
\cite{Moseley06outof}.
Uma análise das aplicações \textit{desktop} da Adobe, relatada em 2006, indicou
que o código que coordena a lógica de manuseio de eventos, \textit{widgets},
e outros componentes da interface gráfica, representa cerca de um terço do código, e
% manuseio, manejo, manipulação
% que o código responsável por componentes da interface gráfica e da lógica de
% coordenação de eventos representa cerca de um terço do código, e
mais da metade dos \textit{bugs} reportados \cite{jarvi2008property}.
Sendo interfaces gráficas com alto grau de interatividade parte inerente
de uma aplicação, seu desenvolvimento e manutenção se tornam um desafio.

%   - observer pattern

% declarative vs imperative
%   - specification (what) vs. execution (how)
%     - Declarative Interaction Design for Data Visualization
%   - modeling vs presentation
%     - Elm
%     - FR Animation
%
% FRP
%  - Outros tipos de software podem ser considerados reativos, como um sistema
%    embarcado que reage a sinais de sensores, ou um sistema distribuído que
%    precisa reagir a mensagens na rede.

<Apresentar FRP como alternativa>

% - Documentar o estado da arte em:
%   - técnologias web
%   - programação assíncrona
% "Este trabalho tem por objetivo apresentar os conceitos, objetivos, tecnologias e
% demais questões envolvidas na abordagem de desenvolvimento de aplicações
% Web conhecida como Ajax. E prover uma aplicação Web de
% georeferenciamento do campus da UFSC utilizando a abordagem Ajax."

<Descrever proposta do projeto>
