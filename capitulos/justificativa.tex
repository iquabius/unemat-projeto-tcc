\section{Justificativa}
\label{sec:justificativa}

Inerentemente \emph{imperativos}, modelos que utilizam \emph{callbacks} como
principal mecanismo de coordenação de eventos são consideradas muito
complexos\cite{ edwards2009coherent, fischer2007tasks, maier2010deprecating,
  reppy1992higher}. Isso se deve a forma desconcertante com que \emph{callbacks}
coordenam alterações no estado do programa. O código da aplicação se torna
difícil de compreender e dar manutenção, podendo ainda ser descrito
coloquialmente como \emph{\enquote{Callback Hell}}
\cite[p.~2]{edwards2009coherent}, como é costumeiro na literatura.

Vale ressaltar que de acordo com \citeonline{moseley06out} o mau gerenciamento
de estado é considerado a principal causa de complexidade em sistemas
contemporâneos, pois impactam a compreensibilidade do programa, e os testes de
software. Uma análise das aplicações \emph{desktop} da Adobe apontou que a
lógica de coordenação de eventos residia em um terço do código e contia metade
dos \emph{bugs} reportados~\cite{jarvi2008property}. Interfaces gráficas
inerentemente dispõem de um alto grau de interativade, tornando seu
desenvolvimento e manutenção um desafio.

\emph{Programação Funcional Reativa (PFR)} é uma alternativa promissora para o
desenvolvimento de sistemas reativos, e tem sido explorada em vários domínios,
como: animação digital~\cite{Elliott-H:1997:Fran},
GUIs~\cite{Czaplicki:2012:Elm}, jogos digitais, robótica, e síntese de música.
PFR permite que aplicações interativas sejam programadas de forma declarativa em
um nível mais elevado de abstração, com código fonte que expressa melhor a
solução implementada. Como resultado, o programa se torna mais compreensível,
mais fácil de dar manutenção e, em geral, menos complexo.

Este trabalho pretende contextualizar a situação atual de como interfaces
orientadas a eventos são implementadas, apresentar os conceitos das abordagens
alternativas, e fornecer implementações de alguns componentes comuns em
interfaces gráficas, com a finalidade de testar e comparar tais abordagens
através do uso de algumas ferramentas (linguagens e/ou \emph{frameworks}).
Atenção especial será dada ao ambiente \emph{web}, ou seja, interfaces de
aplicações utilizadas nos navegadores, e dependendo do levantamento feito,
alguns exemplos poderão ser dados com alguma ferramenta que implemente a PFR
para o ambiente \emph{mobile} e/ou \emph{desktop}.

%%% Local Variables:
%%% mode: latex
%%% TeX-master: "../projeto"
%%% End:
